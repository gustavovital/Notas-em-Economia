\documentclass[11pt,a4paper]{article}
\usepackage[utf8]{inputenc}
\usepackage[portuguese]{babel}
\usepackage[T1]{fontenc}
\usepackage{amsmath}
\usepackage{amsfonts}
\usepackage{amssymb}
\usepackage{graphicx}
\usepackage{lmodern}
\author{Gustavo Vital}
\title{Notas sobre Vetores Auto-regressivos\footnote{Baseado em Walter Enders}}

\begin{document}
\maketitle

Modelos Auto-regressivos são modelos que relacionam sistemas dinâmicos de equações tal que as variáveis das equações são endógenas. De uma outra forma, há dependência contemporânea entre as variáveis, bem como relação com suas defasagens. Considere o sistema bivariado: 
\begin{align} \label{eq1}
y_t &= b_{10} - b_{12}z_t + \gamma_{11}y_{t-1} + \gamma_{12}z_{t-1} + \epsilon_{yt}\\ \label{eq2}
z_t &= b_{20} - b_{21}z_t + \gamma_{21}y_{t-1} + \gamma_{22}z_{t-1} + \epsilon_{zt} 
\end{align}
\noindent
onde assumimos que (i) tanto $y_t$ como $z_t$ são estacionárias; (ii) $\epsilon_{yt}$ e $\epsilon_{zt}$ são ruídos brancos; e (iii) \{$\epsilon_{yt}$\} e \{$\epsilon_{yt}$\} são não relacionados.

\end{document}
\documentclass[11pt,a4paper]{article}
\usepackage[utf8]{inputenc}
\usepackage[portuguese]{babel}
\usepackage[T1]{fontenc}
\usepackage{amsmath}
\usepackage{amsfonts}
\usepackage{xcolor}
\usepackage{amssymb}
\usepackage[left=2.5cm, bottom=2cm, top=2cm, right=2.5cm]{geometry}
\usepackage{graphicx}
\usepackage{lmodern}
\usepackage[linkbordercolor=blue]{hyperref}
\usepackage{setspace}

\author{Gustavo Vital}
\title{Notas sobre Vetores Auto-regressivos\footnote{Baseado em Walter Enders}}

\onehalfspacing

\begin{document}
\maketitle

Modelos Auto-regressivos são modelos que relacionam sistemas dinâmicos de equações tal que as variáveis das equações são endógenas. De uma outra forma, há dependência contemporânea entre as variáveis, bem como relação com suas defasagens. Considere o sistema bivariado: 
\begin{align} \label{eq1}
y_t &= b_{10} - b_{12}z_t + \gamma_{11}y_{t-1} + \gamma_{12}z_{t-1} + \epsilon_{yt}\\ \label{eq2}
z_t &= b_{20} - b_{21}z_t + \gamma_{21}y_{t-1} + \gamma_{22}z_{t-1} + \epsilon_{zt} 
\end{align}
\noindent
onde assumimos que (i) tanto $y_t$ como $z_t$ são estacionárias; (ii) $\epsilon_{yt}$ e $\epsilon_{zt}$ são ruídos brancos; e (iii) \{$\epsilon_{yt}$\} e \{$\epsilon_{yt}$\} são não relacionados. Dizemos que as equações \ref{eq1} e \ref{eq2} constituem um \textbf{vetor auto-regressivo} de ordem 1. A fim de interpretação, o coeficiente $b_{12}$ representa o efeito contemporâneo de $z_t$ em $y_t$ e $\gamma_{12}$ representa o efeito de $z_{t-1}$ em $y_t$.\\

Ainda, as equações \ref{eq1} e \ref{eq2} não podem ser estimadas por OLS visto que $y_t$ tem efeito contemporâneo em $z_t$ e $z_t$ tem efeito contemporâneo em $y_t$. A estimação por OLS seria, então, viesada e os resíduos seriam correlacionados. Felizmente, é possível transformarmos o sistema de equações de forma que este seja mais palpável. Usando notação matricial podemos reescrever o sistema da seguinte forma:

\begin{align*}
\begin{bmatrix}
1 & b_{12} \\
b_{21} & 1 
\end{bmatrix}
\begin{bmatrix}
y_t \\
z_t
\end{bmatrix}
=
\begin{bmatrix}
b_{10}\\
b_{20}
\end{bmatrix}
\begin{bmatrix}
\gamma_{11} & \gamma_{12} \\
\gamma_{21} & \gamma_{22}
\end{bmatrix}
\begin{bmatrix}
y_{t-1}\\
z_{t-1}
\end{bmatrix}
+
\begin{bmatrix}
\epsilon_{yt}\\
\epsilon_{zt}
\end{bmatrix}
\end{align*}
\noindent
ou ainda:
\begin{align} \label{bvar}
Bx_t = \Gamma_0 + \Gamma_1 x_{t-1} + \epsilon_t
\end{align}
\noindent
onde:
\begin{align*}
B= \begin{bmatrix}
1 & b_{12} \\
b_{21} & 1 
\end{bmatrix}, x_t = \begin{bmatrix}
y_t \\
z_t
\end{bmatrix}, \Gamma_0 = \begin{bmatrix}
y_t \\
z_t
\end{bmatrix}, \Gamma_1 = \begin{bmatrix}
y_t \\
z_t
\end{bmatrix}, \epsilon_t = \begin{bmatrix}
\epsilon_{yt}\\
\epsilon_{zt}
\end{bmatrix}
\end{align*}
\noindent
pré multiplicando \ref{bvar} por $B^{-1}$, obtemos o VAR em sua forma reduzida:
\begin{align}
x_t = A_0 + A_1 x_{t-1} + e_t
\end{align}
\noindent
onde $A_0 = B^{-1}\Gamma_0$, $A_1 = B^{-1}\Gamma_1$, e $e_t = B^{-1}\epsilon_t$

\end{document}
\documentclass[11pt,a4paper]{article}
\usepackage[utf8]{inputenc}
\usepackage[portuguese]{babel}
\usepackage[T1]{fontenc}
\usepackage{amsmath}
\usepackage{amsfonts}
\usepackage{xcolor}
\usepackage{amssymb}
\usepackage{graphicx}
\usepackage[left=2.5cm, bottom=2cm, top=2cm, right=2.5cm]{geometry}
\usepackage{lmodern}
\usepackage[linkbordercolor=blue]{hyperref}

\author{Gustavo Vital\thanks{Faculdade de Economia do Porto -- FEP. Email: gustavoovital@id.uff.br}}
\title{Notas sobre o modelo de Ramsey-Cass-Koopman}

\begin{document}
\maketitle

Modelos de crescimento econômico frequentemente são revisitados e revisados. Em termos do \textit{mainstream}, o modelo de Solow (mesmo responsável pelo prêmio Nobel) introduz o conceito de exogeneidade ao crescimento de uma nação. Levando em consideração -- na sua forma mais básica -- tecnologia; estoque de capital; e trabalho; Solow apresenta de forma sucinta seu modelo de crescimento. O modelo, entretanto, peca em alguns fatores considerados cruciais para o melhor entendimento de como se dá o crescimento de uma economia. Sua premissa de exogeneidade quanto a choques tecnológicos também é contestável, visto que tecnologia acontece fundamentalmente devido a acumulo de conhecimento, educação.

\section{O Modelo de Ramsey-Cass-Koopman}  

O modelo de Ramsey-Cass-Koopman se apresenta como um modelo de horizonte infinito. A vantagem fundamental é explicar de forma matemática o processo de decisão dos agentes, de forma microfundamentada. A compreensão desse modelo é essencial para a compreensão da teoria de \textit{Real Business Cycle} (RBC).

\subsection{Premissas do Modelo}

Considera-se que existe um número infinito de famílias idênticas e o tamanho de cada família cresce a uma taxa $n$. A equação que descreve o crescimento da população é dada por:
\begin{align*}
\frac{dL}{dn} = nL
\end{align*}
\noindent
integrando de ambos os lados, ficamos com:
\begin{align*}
\int_{L_0}^{L}\frac{dL}{L} &= \int_0 ^t ndt 
\end{align*}
\noindent
o que nos dá:
\begin{align*}
\log \left(\frac{L}{L_0}\right) &= nt 
\end{align*}
\noindent
aplicando o exponencial:
\begin{align} \label{eq:cresc}
L(t) = L_0 e^{nt}
\end{align}
A equação \eqref{eq:cresc} representa o crescimento populacional da economia onde $L(t)$ representa a população total da economia; $L_0$ a população inicial no momento analisado e $t$ o tempo percorrido. Se considerarmos $H$ como o número de famílias e $\bar{C}$ o consumo total da economia. O consumo per capita será representado por:
\begin{align*}
\frac{\bar{C}(t)}{L(t)}
\end{align*} 
\noindent
e o consumo por membro família $(C(t))$ será dado por:
\begin{align} \label{eq:fam}
C(t) = \frac{\bar{C}(t)}{\left(\frac{L(t)}{H}\right)}
\end{align}
\noindent
de onde tiramos que o consumo total pode ser escrito como:
\begin{align*}
\bar{C}(t) = C(t)\frac{L(t)}{H}
\end{align*}

\subsection{Utilidade}
Tomando por base um modelo microfundamentado, é essencial que tomemos em consideração a utilidade dos agentes. Representando a utilidade de cada membro da família por $u(C(t))$, podemos generalizar e representar a utilidade da população no período $t$ por:
\begin{align*}
U(\bar{C}(t)) = u(C(t))\frac{L(t)}{H}
\end{align*} 
\noindent
A utilidade, então, da vida dos indivíduos pode ser representada por:
\begin{align}
U = U(\bar{C}(0)) + \beta U(\bar{C}(1)) + \beta^2 U(\bar{C}(2)) + \dots + \beta^t U(\bar{C}(t)) 
\end{align}
\noindent
em que $\beta$ representa o fator de desconto intertemporal da utilidade. O problema da equação acima é levar em consideração que o tempo é uma variável discreta. O problema pode ser representado de outra maneira, tomando por base que o tempo é uma variável contínua:
\begin{align} \label{eq:cont}
U = \int_0 ^t \beta^t U(\bar{C}(t))dt
\end{align}
\noindent
podemos reescrever \eqref{eq:cont} de forma que considere a utilidade por membro da família. Unindo \eqref{eq:fam} e \eqref{eq:cont} e considerando um horizonte infinito de tempo, temos:
\begin{align} \label{eq:ltu}
U = \int_0 ^\infty \beta^t u(C(t))\frac{L(t)}{H} dt
\end{align}
\noindent
a equação \eqref{eq:ltu} é chamada de \textit{lifetime utility}, e demonstra que a geração no período $t$ é seguida de seus descendentes, com transferências intergeracionais baseadas no altruísmo.\\

Por mais que o modelo de Ramsey-Cass-Koopman não considere expectativas e incertezas, a partir da especificação da função de utilidade é possível inferir sobre determinadas preferências dos agentes. Por exemplo, ao considerarmos que a função utilidade assume um formato CRRA\footnote{$u(c) = \frac{c^{1-\theta}}{1-\theta}$}, quando o parâmetro $\theta$ se aproxima de zero a famílias estão dispostas a aceitarem oscilações no consumo afim de tirarem vantagem da diferença entre a taxa de desconto e a taxa de retorno sobre a poupança. Considerando uma função utilidade de formato CRRA, temos que a \textit{lifetime utility} pode ser expressa por:
\begin{align*} 
U = \int_0 ^\infty \beta^t \left(\frac{c^{1-\theta}}{1-\theta}\right)\frac{L(t)}{H} dt
\end{align*}

\subsection{Produção e Estoques}

A função de produção no modelo de Ramsey-Cass-Koopman se assemelha a função de produção no modelo de Solow. A tecnologia, entretanto, é tratada como uma variável endógena. Definimos de forma genérica a função de produção:
\begin{align*}
Y(t) = F(A(t), K(t), L(t))
\end{align*}
\noindent
em que A(t) representa a tecnologia, K(t) o estoque de capitais, e L(t) o trabalho. No modelo, somente os níveis iniciais das variáveis são considerados exógenos e crescem a uma taxa constante, tal que:
\begin{align*}
\dot{L}(t) &= nL(t) \\
\dot{A}(t) &= gA(t)
\end{align*} 
\noindent
consideramos, também, que a função de produção possui retornos constantes de escala. Isso é $F(\alpha K, \alpha AL)= \alpha F(K,AL)$. Assim, assumindo retorno constante de escala e dividindo a função de produção por AL e sendo $k = \frac{K}{AL}$, temos:
\begin{align*}
F\left(\frac{K}{AL}, 1\right) = \frac{1}{AL}F(K, AL) \Rightarrow \underbrace{F(k, 1)}_{f(k)} = \frac{Y}{AL}
\end{align*}
\noindent
$f(k)$ é chamada de \textbf{função de produção na sua forma intensiva}

\section{Comportamento das Firmas}

A premissa inicial para o comportamento das firmas é que, seja em qual for a parte do tempo, a firmas utilizam estoque de trabalho e capital e como pagamento utilizam seus produtos marginais de forma que vendem sua produção resultante. Além disso, é considerado lucro zero, visto que a economia se encontra em estado de competitividade. Isso significa que: a remuneração do capital é a própria produtividade marginal do capital $f'(k)$. Ainda, como não existe depreciação, a taxa de retorno real sobre o capital iguala seus ganhos por unidade de tempo:
\begin{align}
r(t) = f'(k(t))
\end{align}
\noindent
como a firma remunera de acordo com a produtividade marginal:
\begin{align}
W(t) = A(t)[f(k(t)) - k(t)f'(k(t))]
\end{align}
\noindent
então o salário por unidade efetiva é:
\begin{align}
w(t) = f(k(t)) - k(t)f'(k(t))
\end{align}

\end{document}